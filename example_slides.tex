\documentclass[pdf]{beamer}
%\mode<presentation>{}

\usepackage{amssymb,amsmath,amsthm,enumerate}
\usepackage[utf8]{inputenc}
\usepackage{array}
\usepackage[parfill]{parskip}
\usepackage{graphicx}
\usepackage{caption}
\captionsetup[figure]{labelformat=empty}
\usepackage{subcaption}
\usepackage{amsmath}
\usepackage{bm}
\usepackage{amsfonts,amscd}
%\usepackage{gensymb}
\usepackage[]{units}
\usepackage{listings}
\usepackage{multicol}
\usepackage{tcolorbox}
\usepackage{physics}

%new commands
\newcommand{\der}[2]{\frac{d#1}{d#2}}
\newcommand{\nder}[3]{\frac{d^#1 #2}{d #3 ^ #1}}
\newcommand{\pder}[2]{\frac{\partial #1}{\partial #2}}
\newcommand{\npder}[3]{\frac{\partial ^#1 #2}{\partial #3^#1}}
\newcommand{\sentencelist}{def}
\newcommand{\overbar}[1]{\mkern 1.5mu\overline{\mkern-1.5mu#1\mkern-1.5mu}\mkern 1.5mu}
\newcommand{\lined}{\overbar}
\newcommand{\perm}[2]{{}^{#1}\!P_{#2}}
\newcommand{\comb}[2]{{}^{#1}C_{#2}}
\newcommand{\intall}{\int_{-\infty}^{\infty}}
\newcommand{\Var}[1]{\text{Var}\left(#1\right)}
\newcommand{\E}[1]{\text{E}\left(#1\right)}
\newcommand{\define}{\equiv}
\newcommand{\diff}[1]{\mathrm{d}#1}
\newcommand{\empy}[1]{{\color{darkorange}\emph{#1}}}
\newcommand{\empr}[1]{{\color{cardinalred}\emph{#1}}}


\theoremstyle{remark}
\newtheorem*{remark}{Remark}
\theoremstyle{definition}

\newcommand{\examplebox}[2]{
\begin{tcolorbox}[colframe=darkcardinal,colback=boxgray,title=#1]
#2
\end{tcolorbox}}

\newcommand{\eld}[1]{\frac{d}{dt}(\frac{\partial L}{\partial \dot #1}) - \frac{\partial L}{\partial #1}=0}
\newcommand{\euler}[1]{\frac{\partial L}{\partial #1}-\frac{d}{dt}(\frac{\partial L}{\partial \dot #1})}
\newcommand{\eulerg}[1]{\frac{\partial g}{\partial #1}-\frac{d}{dt}(\frac{\partial g}{\partial \dot #1})}
\newcommand{\divg}[1]{\nabla\cdot #1}
\newcommand{\prob}[1]{P(#1\vert I)}



\usetheme{Stanford} 
\def \i  {\item}
\def \ai {\item[] \quad \arrowbullet}
\newcommand \si[1]{\item[] \quad \bulletcolor{#1}}
\def \wi {\item[] \quad $\ \phantom{\Rightarrow}\ $}
\def \bi {\begin{itemize}\item}
\def \ei {\end{itemize}}
\def \be {\begin{equation*}}
\def \ee {\end{equation*}}
\def \bie {$\displaystyle{}
\def \eie {{\ }$}}
\def \bsie {\small$\displaystyle{}
\def \esie {{\ }$}\normalsize\selectfont}
\def \bse {\small\begin{equation*}}
\def \ese {\end{equation*}\normalsize}
\def \bfe {\footnotesize\begin{equation*}}
\def \efe {\end{equation*}\normalsize}
\renewcommand \le[1] {\\ \medskip \lefteqn{\hspace{1cm}#1} \medskip}
\def \bex {\begin{example}}
\def \eex {\end{example}}
\def \bfig {\begin{figure}}
\def \efig {\end{figure}}
\def \btheo {\begin{theorem}}
\def \etheo {\end{theorem}}
\def \bc {\begin{columns}}
\def \ec {\end{columns}}
\def \btab {\begin{tabbing}}
\def \etab {\end{tabbing}\svneg\svneg}
\newcommand \col[1]{\column{#1\linewidth}}
\def\vneg  {\vspace{-5mm}}
\def\lvneg {\vspace{-10mm}}
\def\svneg {\vspace{-2mm}}
\def\tvneg {\vspace{-1mm}}
\def\vpos  {\vspace{5mm}}
\def\lvpos {\vspace{10mm}}
\def\svpos {\vspace{2mm}}
\def\tvpos {\vspace{1mm}}
\def\hneg  {\hspace{-5mm}}
\def\lhneg {\hspace{-10mm}}
\def\shneg {\hspace{-2mm}}
\def\thneg {\hspace{-1mm}}
\def\hpos  {\hspace{5mm}}
\def\lhpos {\hspace{10mm}}
\def\shpos {\hspace{2mm}}

\logo{\includegraphics[height=0.4in]{./style_files_stanford/SU_New_BlockStree_2color.png}}



\title[Buzz Words, maybe Clever Abbreviations]{Some Cool Sounding Buzzwords}
\subtitle{Less Buzzy and Somewhat More Explanaotry Subtitle}


\beamertemplatenavigationsymbolsempty

\begin{document}



\author[S. Cheong, Stanford]{
	\begin{tabular}{c} 
	\Large
	Sanha Cheong\\
    \footnotesize \href{mailto:sanha@stanford.edu}{sanha@stanford.edu}
\end{tabular}
\vspace{-4ex}}

\institute{
	\includegraphics[height=0.4in]{./style_files_stanford/SU_New_BlockStree_2color.png}\\
	Department of Physics\\
	Stanford University}

\date{\today}

\begin{noheadline}
\begin{frame}\maketitle\end{frame}
\end{noheadline}



\begin{frame}{Goals}
The main goals for this slide:
\begin{itemize}
	\item This is just to show you how this \empy{template} works
	\item There are two `emphasize' functions used to highlight \& italicize texts
	\begin{itemize}
		\item `\textbackslash empy' does \empy{this} and `\textbackslash empr' does \empr{this}.
	\end{itemize}
	\item The colors in this template is selected from the official Stanford Identity website: \href{https://identity.stanford.edu/color.html}{https://identity.stanford.edu/color.html}
	\item Note that hyperlinks, by default, are not highlighted. Of course, you can change this: e.g., \empr{\href{https://github.com/sanhacheong/stanford_beamer_template}{https://github.com/sanhacheong/stanford\_beamer\_template}}
\end{itemize}
\end{frame}



\begin{frame}{Example}
A (hopefully) useful function in this \LaTeX~template is:
\begin{center}
	\textbackslash examplebox\{ExampleTitle\}\{ExampleContents\}
\end{center}
which does this:
\examplebox{Example of the Command \textbackslash examplebox}{
This is what it does. Pretty self-explanatory, isn't it?

Given the color them, I \empr{recommend} using \textbackslash empr inside of examplebox. The \textbackslash empy command does not look \empy{that} good.
}
\end{frame}



\begin{frame}{References}
\begin{thebibliography}{3}
	\bibitem{myself}
	S.~Cheong. \empr{\href{https://github.com/sanhacheong/stanford_beamer_template}{https://github.com/sanhacheong/stanford\_beamer\_template}}. {GitHub}, August 2017.
\end{thebibliography}
\end{frame}

\end{document}